\documentclass{article}

\usepackage[utf8]{inputenc}
\usepackage{authblk}
\usepackage[dvipsnames]{xcolor}
\usepackage{amsmath}

\title{Code Style}
\author{Elizaweta Chernova ; IA-031 ;}
\affil{SibSUTIS, email: elizaweta\_02@mail.ru .}
\date{January 2022}

\begin{document}

\maketitle

\section{Introduction}
This document contains my code style guide .
\section{Programming languages used}.
\textbf{\large{C++\cite{CPP}}}\\
\section{Code-Style for C++} 
\textbf{\large{Function Declarations and Definitions\cite{CPP1}}}\\
Return type on the same line as function name, parameters on the same line. \\

\begin{tabular}{|p{300}|}
\hline\\
ReturnType FunctionName(type1 name1, type2 name2)\ \{ \\ 
\ \ \ \ DoSomething(); \\
\ \ \ \ ...\\ 
\} \\ \\
\hline
\end{tabular}\\ \\ \\ 
\textbf{\large{Function Calls}}\\
Write the call all on a single line, wrap the arguments at the parenthesis\\ 

\begin{tabular}{|p{300}|}
\hline\\
int result = DoSomething(argument1, argument2, argument3); \\ \\
\hline
\end{tabular}\\ \\ \\ \\
\textbf{\large{Comment Style\cite{CPP3}}}\\
The syntax uses // or /* */ , but /* */ is used less often, only when you need to encode a large section of code or description. \\ \\

\begin{tabular}{|p{300}|}
\hline\\
//this section calculates the factorial\\ \\
/*this section calculates the factorial*/\\ \\
\hline 
\end{tabular}\\ \\ \\
\textbf{\large{Whitespaces}}\\
Open braces should always have a space before them and two spaces before end-of-line comments. \\ \\

\begin{tabular}{|p{300}|}
\hline\\
void function(int b) \{  // function for ...\\\\
\hline 
\end{tabular}\\ \\ \\
Semicolons usually have no space before them.\\
Assignment operators always have spaces around them.\\ \\

\begin{tabular}{|p{300}|}
\hline\\
int i = 0;\\\\
\hline 
\end{tabular}\\ \\ \\
Other binary operators usually have spaces around them.\\
Parentheses should have no internal padding.\\ \\

\begin{tabular}{|p{300}|}
\hline\\
x = a * b + (c / d);\\\\
\hline 
\end{tabular}\\ \\ \\ \\
No spaces separating unary operators and their arguments.\\ \\

\begin{tabular}{|p{300}|}
\hline\\
x = -5;\\
x++;\\
if (x\ \&\&\ \!y)\\
  ...\\\\
\hline 
\end{tabular}\\ \\ \\
\textbf{\large{Variable Names}}\\
Ordinarily, functions should start with a capital letter and have a capital letter for each new word.\\ \\

\begin{tabular}{|p{300}|}
\hline\\
AddTableName()\\
DeleteString()\\
OpenNewFile()\\\\
\hline 
\end{tabular}\\ \\ \\
Variables declared const are named with a leading "k" followed by mixed case.\\ \\

\begin{tabular}{|p{300}|}
\hline\\
const int kDaysInAWeek = 7;\\\\
\hline 
\end{tabular}\\ \\ \\
Each variable is declared on a new line.\\
Variables can be enumerated in a string if they are of the same type.\\

\begin{tabular}{|p{300}|}
\hline\\
int a;\\
float b;\\
char d, e, f;\\\\
\hline 
\end{tabular}\\ \\ \\
\textbf{\large{Conditional statements\cite{CPP2}}}\\
Put one space between the if and the opening parenthesis, and between the closing parenthesis and the curly brace (if any)\\
Use curly braces for the controlled statements following if, else if and else. Break the line immediately after the opening brace, and immediately before the closing brace. A subsequent else, if any, appears on the same line as the preceding closing brace, separated by a space.\\

\begin{tabular}{|p{300}|}
\hline\\
if (condition) \{\\                   
\ \ DoOneThing();\\                    
\ \ DoAnotherThing();\\
\} else if (int a = f(); a != 3) \{\\   
  DoAThirdThing(a);\\
\ \ \} else \{\\
\ \ DoNothing();\\
\ \ \}\\ \\
\hline 
\end{tabular}\\ \\ \\
\textbf{\large{Loops Statements}}\\
The closing curly brace is always written in line with for/while\\
Braces are optional for single-statement loops.\\ \\

\begin{tabular}{|p{300}|}
\hline\\
for (int i = 0; i < kSomeNumber; ++i) \{ \\
\ \ printf("my dog is angry"); \\
\} \\
for (int i = 0; i < kSomeNumber; ++i) \\
\ \ printf("hello world\n"); \\
\\
while (1) \{\\
\ \ DoSomething();\\
\} \\
\hline
\end{tabular}\\ \\ \\ 
\textbf{\large {Assigning a text value to a string}}\\ \\
Text is written to a variable of type string with quotation marks.\\
There is no space between the word and the quotation marks.\\ \\

\begin{tabular}{|p{300}|}
\hline\\
string str = "Hello world";\\ \\
\hline
\end{tabular}\\ \\ \\
\textbf{\large {Output on display}}\\ \\
Screen output with the cout statementю
Text are always preceded by a space on both sides \ll \\ \\

\begin{tabular}{|p{300}|}
\hline\\
cout \ll "a =" \ll a \ll endl ;\\ \\
\hline 
\end{tabular}\\ \\ \\ 




\section{Conclusion}
\large{I learned how to use Latex commands and learned how to work with them, I also made Code style, which will help to beautifully and correctly format the program code}\\
\hline
\begin{thebibliography}{4}
\bibitem{CPP}
ISO/IEC 14882 Programming languages — C++.
\bibitem{CPP1}
Google C++ Style Guide : [electronic resource].
URL : https://google.github.io/styleguide/cppguide.html
\bibitem{CPP2}
C++ Code Design Guide from Stanford University. : [electronic resource]. 
URL : https://tproger.ru/translations/stanford-cpp-style-guide/
\bibitem{CPP3}
Lesson number 9. Comments. : [electronic resource]. 
URL : https://ravesli.com/urok-9-kommentarii-v-s/#toc-2


\end{thebibliography}

\end{document}